\documentclass[11pt,a4paper]{moderncv}

\moderncvtheme[blue]{classic}

\usepackage[utf8]{inputenc}
\usepackage[scale=0.88]{geometry}

\firstname{Pankaj}
\familyname{\mbox{Kumar}}

\vspace*{-1.0cm}

\title{\small Explainable and Causal AI $\star$ Python Developer $\star$ Git $\star$ Physical Modeling}
\mobile{+91~7061255826}
\email{pankaj.kmr1990@gmail.com}
\homepage{pankajkarman.github.io}
\social[linkedin]{pankajkmr1990}
\social[github]{pankajkarman}

% the ConTeXt symbol
\def\ConTeXt{%
  C%
  \kern-.0333emo%
  \kern-.0333emn%
  \kern-.0667em\TeX%
  \kern-.0333emt}

% command and color used in this document, independently from moderncv 
\definecolor{see}{rgb}{0.5,0.5,0.5}% for web links
\newcommand{\up}[1]{\ensuremath{^\textrm{\scriptsize#1}}}% for text subscripts
\renewcommand{\labelitemi}{$\circ$}

%----------------------------------------------------------------------------------
%            content
%----------------------------------------------------------------------------------
\begin{document}
\maketitle
\vspace*{-1.5cm}

\section{\textsc{Education}}
%    \cventry{years}{degree/job title}{institution/employer}{localization}{grade}{description}
\cventry{2022}{Doctor of Philosophy, Atmospheric Chemistry and Machine Learning}{}{}{}{Indian Institute of Technology Kharagpur (IN)}
\cventry{2017}{Master of Technology, Earth System Science and Technology}{}{}{}{Indian Institute of Technology Kharagpur (IN)}
\cventry{2012}{Bachelor of Engineering, Mechanical Engineering}{}{}{}{Birla Institute of Technology, Mesra (IN)}

\section{\textsc{Research Experience}}
		\cventry{2022 - Present}{Post-Doctoral Researcher}{Aerosol and Reactive Tracer Modelling, KIT Germany}{}{}{
			{\footnotesize 
				\begin{itemize}
				\item Implementing ML based emulation and parameterization schemes for ICON-ART model.
				\item Developing mineral dust pre-processor for ICON modeling system.
				\item Authoring user-friendly post-processing library for ICON-ART in python.	
				\end{itemize}
			}
		}
		\cventry{2017 - 2022}{Research Scholar, PhD}{ATMOS Lab, IIT Kharagpur}{}{}{
			{\footnotesize 
				\begin{itemize}
				\item Developed bias-correction library in python (\href{https://github.com/pankajkarman/bias_correction}{\underline{>24k downloads till now}}).
				\item Developed receptor models for pollutant source detection using back-trajectories in python (\href{https://github.com/pankajkarman/HyTraj}{\underline{>8k downloads}}).
				\item Implemented clustering of air-parcel trajectories using wavelet features for transportation pathways analysis.				
				\item Performed self-organising map based clustering of tropospheric ozone profiles and their trend analysis using Bayesian dynamic linear model and multivariate linear regression.\item Conducted causal analysis of tropospheric ozone to identify the geophysical drivers of observed variability.
				\item Investigated Land Use Land Cover change over North-East India using Google Earth Engine and Random forest based classification.
				\item Developed a sequence-to-sequence autoencoder to extract features from variable length trajectories.
				\item Simulated global atmospheric chemistry using GEOS-Chem at Pratyush, India's fastest supercomputer.	
				\end{itemize}
			}
		}
		\cventry{2016 - 2017}{Research Assistant, MTech}{ATMOS Lab, IIT Kharagpur}{}{}{
			{\footnotesize 
				\begin{itemize}
				\item Estimated rainfall using preliminary Doppler Weather radar data for Kolkata region using python.
				\item Investigated freezing and shape transformation of water droplet numerically using MATLAB.	
				\end{itemize}
			}
		}

\section{\textsc{Technical Skills}}
		\cvitem{\textbullet}{\textbf{Data Analytics}: Bayesian inference, Machine Learning, Causal Discovery and Inference}
		\cvitem{\textbullet}{\textbf{Physical Modeling}: HySPLIT, WRF, GEOS-Chem, ICON-ART}
		\cvitem{\textbullet}{\textbf{Programming}: Python, Fortran, JavaScript, MATLAB, Bash, Git}
		\cvitem{\textbullet}{\textbf{Markup Languages}: \LaTeX, Markdown, HTML/CSS}

\section{\textsc{Publications}}
		\cvitem{\textbullet}{Rahul Kashyap, Jayanarayanan Kuttippurath and \textbf{Pankaj Kumar}, \textit{Browning of vegetation in efficient carbon sink regions of India during the past two decades is driven by climate change and anthropogenic intrusions}, {\color{see} \href{https://doi.org/10.1016/j.jenvman.2023.117655}{\underline{Journal of Environmental Management}, 2023.}}}
		\cvitem{\textbullet}{\textbf{Pankaj Kumar}, Jayanarayanan Kuttippurath and Adway Mitra, \textit{Causal discovery of drivers of surface ozone variability in Antarctica using a deep learning algorithm}, {\color{see} \href{https://pubs.rsc.org/en/content/articlelanding/2022/EM/D1EM00383F}{\underline{RSC Environmental Science: Processes \& Impacts}, 2022.}}}
                 \cvitem{\textbullet}{\textbf{Pankaj Kumar}, Jayanarayanan Kuttippurath, Peter von der Gathen, Irina Petropavlovskikh, Bryan Johnson, Audra McClure-Begley, Paolo Cristofanelli, Paolo Bonasoni, Maria Elena Barlasina, and Ricardo Sánchez, \textit{The increasing surface and tropospheric ozone in Antarctica and their possible drivers}, {\color{see} \href{https://pubs.acs.org/doi/10.1021/acs.est.0c08491}{\underline{Environmental Science \& Technology}, 2021.}}}
                 
		\cvitem{\textbullet}{J. Kuttippurath, \textbf{P. Kumar}, P. J. Nair, P C Pandey, \textit{\href{}{Emergence of ozone recovery evidenced by reduction in the occurrence of Antarctic ozone loss saturation,}} {\color{see} \href{https://www.nature.com/articles/s41612-018-0052-6}{\underline{npj Climate and Atmospheric Science}, 2018.}}}
	
\vspace*{-0.1cm}		
\section{\textsc{Achievements}}
		% \cvitem{LHS}{RHS}
		\cvitem{\textbullet}{\href{https://pankajkarman.github.io/news/index.html}{\underline{Multiple researcher articles}} covered by reputed national/international media like \href{https://www.thehindu.com/sci-tech/science/wettest-place-on-earth-sees-decreasing-trend-in-rainfall/article61746996.ece}{\underline{the Hindu}}.}
		\cvitem{\textbullet}{Open-source libraries developed by me has crossed 30k downloads.}

\end{document}
%% end of file `page.tex'.
